\chapter{词法分析}

词法分析跟语法分析有很多相同的地方,工作在不同的抽象层次,所以有很多地方可以进行类比理解。其一:在词法分析中,分析的基本单元是字母表里面的符号;而在语法分析中,分析的基本单元是词法分析所向上提供的词法单元,可以知道,词法单元是由字母表中的字母所组成的,这说明语法分析工作在比词法分析更高的层次上。其二:在词法分析中,对于分析机理的描述采用的是正则表达式,而在语法分析中,对分析机理的描述是采用语法描述,这两者的本质其实都是状态机,词法分析中,状态转移接受字母表的基本符号作为输入,而在语法分析中,状态转移接受词法单元作为输入,它们的本质都是一样的,只不过工作在不同抽象层次上面;其三:在词法分析中,分析的目标是得出词法单元,而在语法分析中,分析的目标是句子,而句子正是由词法单元组成,再一次说明工作所在的抽象层次不一样。由此我们可以对比着对两者进行理解,从而找到其共性。

在词法分析中,输入是字符串,而输出是词法单元流。有两种主要的分析流程可以达到这样的目的,其中第一种为:

\begin{enumerate}
    \item 为中序正则表达式添加连接符。
    \item 将中序正则表达式转化为后序正则表达式。
    \item 根据后序正则表达式建立NFA。
    \item 将NFA转化为DFA。
    \item 将DFA的状态最小化。
\end{enumerate}

第二种方式为:

\begin{enumerate}
    \item 为中序正则表达式添加连接符。
    \item 将中序正则表达式转化为后序正则表达式。
    \item 根据后序正则表达式建立抽象语法树。
    \item 计算抽象语法树的$nullabe$,$firstpos$,$lastpos$以及$followpos$。
    \item 根据得出的数据直接构造DFA。
\end{enumerate}

由上面的步骤可以看出,两种方法都必须要先对正则表达式进行处理,鉴于时间关系,我没能把两种方法都实现,于是只实现了第一种方法。

\section{正则表达式}

正则表达式的实现是一个复杂的问题。在JavaScript里面,有现成的正则表达式工具,但基于我这次项目的目的,我打算自己实现正则表达式。在实现的过程中遇到了不少的问题。

首先,正则表达式分为基本正则表达式以及扩展正则表达式。在不同编程语言里面所提供的正则表达式工具都是强大的扩展正则表达式,其中具有很多的操作符以及诸多强大的功能。而基本的正则表达式只有三种操作:

\begin{center}\begin{tabular}{c|c}
    $\bf{Operations}$ & $\bf{Definitions}$ \\
    \hline
    $Union$          & $L \cup M = \{s|s \in L \| s \in M\}$  \\
    $concatenation$  & $LM = \{st|s \in L \&\& s \in M\}$  \\
    $Kleene closure$ & $L^* = \cup^\infty_{i=0} L^i$  \\
\end{tabular}\end{center}

这三种操作已经可以完全表示扩展表达式中的其他操作,所以为了实现上的简洁,我实现的是基本正则表达式。

在实现过程中遇到的另外一个重要的问题是连接符的问题,由于在输入正则表达式的时候是不会显式输入连接符,这给正则表达式的分析带来了很大的困难。比如考虑正则表达式$(a|b)*abb$,如果直接对其进行中序转后序的操作,我们得到的将会是$ab|*abb$,然而实际上这样做是不行的,因为连接操作并没有在这里体现出来,如果就这样直接进行NFA的构建,那么将会得到一个错误的自动机。解决的办法有两个,一个是在中序正则表达式加上连接符以后再转化为后序正则表达式;二是在中序正则表达式转化为后序正则表达式之后再添加连接符;考虑到括号为我们提供的诸多信息,我的实现是先为中序的正则表达式添加上连接符,如上面的正则表达式,添加上连接符以后变成:$(a|b)*\sim a\sim b\sim b$(这里假设用$\sim$来表示连接符)。这样处理以后,转化出来的后序正则表达式为:$ab|*a\sim b\sim b\sim$,这才是正确的后序转换,其后构建状态机才不会出错。

解决了这个问题以后,下一个问题随即出现:应该在什么地方添加连接符呢?之前正是为了这个原因才选择基本正则表达式的,其可能出现的情况比较少,下面可以来穷举一下几种需要添加连接符的情况:

\begin{eqnarray*}
    Char \quad Char & \rightarrow & Char \sim Char \\
    Char \quad (    & \rightarrow & Char \sim ( \\
    ) \quad Char    & \rightarrow & ) \sim Char \\
    * \quad Char    & \rightarrow & * \sim Char \\
    * \quad (       & \rightarrow & * \sim ( \\
    ) \quad (       & \rightarrow & ) \sim ( \\
    any \quad endmarker & \rightarrow & any \sim endmarker \\
\end{eqnarray*}

上面的情况应该也很容易理解,就拿第一种情况来说,如果两个字母表中的字符粘连在一起,那么它们之间必定有一个连接符,因为没有0元操作符可以连接两个字符。又比如右括号和字符之间必定有连接符,右括号预示着一个部分的结束,它后面除非是结束,不然一定需要一个左结合的运算符将其与后面的部分连接,所以这种情况也必须要添加连接符。其余的如此类推。   

解决了这些问题以后,就可以思考正则表达式的数据结构了。首先正则表达式是一个字符串,故可以用字符串来表示正则表达式,考虑到正则表达式需要时时查询字母表,所以有必要把字母表也记录下来,故正则表达式的数据结构可以简单地表示为:

\begin{verbatim}

ReExpression:
    reExp    -> Array(Char) # 正则表达式
    alphabet -> Array(Char) # 正则表达式所表示语言的字母表

\end{verbatim}

有了正则表达式的基本模型以后,可以根据上面的讨论很轻易地写出添加连接符的$insertConcatenation$函数,其伪代码如下:

\begin{verbatim}

function insertConcatenation:
  left  = reExp[0]
  right = reExp[1]

  while (right != endmarker):
    if left and right satisfy the conditions above:
      reExp.insert('~')

    left moves forward
    right moves forward

\end{verbatim}

这里在实现的时候有一个问题需要注意,那就是插入连接符的动作其实是对原字符串进行了修改,如果整个过程都在原字符串上面操作,那么在每次插入之后都必须显式将下标向前移动,以配合新插入的连接符。如果上面的步骤是在一个全新的字符串上面进行操作,那么就不会出现下标错位的问题。


\section{中序表达式与后序表达式}

在处理了连接符的问题以后,就可以开始考虑怎样把中序表达式转化为后序表达式了。首先必须明白为什么要将中序表达式转化为后序表达式。因为中序表达式具有二义性,而后序表达式没有,再者,在中序表达式中,括号仅仅起着分组的作用,并没有任何的运算功能,所以没有必要去保留。在后序表达式中,运算符均在其相关操作数的后方,所以可以直接根据操作符号的元数来进行操作数个数的选取。对于这一点来说,中序表达式是不能做到的,因为在运算符后方仍有与其相关的操作数。

了解了中序表达式到后序表达式的重要性,我们还得明白整个转换过程的机理。由于中序表达式具有二义性,所以在转换的时候需要作出一些规定。看如下算术表达式:$a + b \times c$,其转化为后序表达式以后可以有两种形式:

\begin{enumerate}
    \item $a b + c \times$
    \item $a b c \times +$
\end{enumerate}

可以看到,对于第一种转换方法,是先将$a$于$b$相加,然后再与$c$相乘,而对于第二种转换方法,则是先将$b$于$c$相乘,然后再与$a$相加,这样就说明了转换为后序表达式的时候会对运算符的优先级有所依赖,由此,在转换的时候需要根据运算符的优先级进行转换,使得转换出来的结果符合我们所需要的语义。那上面的例子来说,即我们需要的结果是第二种转换方式。

根据正则表达式的语义,我们可以人为表示出各个运算符的优先级,具体看下表:

\begin{center}\begin{tabular}{|c|c|}
    \hline
    \bf{Operator} & \bf{Precedency} \\ \hline
    $($      & $0$        \\ \hline
    $)$      & $0$        \\ \hline
    $*$      & $9$        \\ \hline
    $\sim$   & $8$        \\ \hline
    $|$      & $7$        \\ \hline
\end{tabular}\end{center}

其实这个优先级的数字并没有太多的含义,只要数字之间能够体现其优先级的先后关系即可,这里对数字的选取其实也为了后面对更多运算符的添加预留了空间,所以才从7开始。之所以将括号的优先级设到最低,是因为它们仅仅是为了分组而已,并没有运算功能。下面给出中序表达式转成后序表达式的伪代码:

\begin{verbatim}

function toPosfix(precedencyTable, dimensionTable):
  opStack = {}
  postfix = {}

  while True:
    # 正则表达式的末尾
    if symbol is endmarker:
      while opStack is not empty:
        postfix.push(opStack.pop())
      break

    # 遇到左括号
    if symbol is ``('':
      opStack.push(symbol)

    # 遇到右括号
    else if symbol is ``)'':
      while !opStack.empty and opStack.top != ``('':
        postfix.push(opStack.pop)
      opStack.pop

    # 遇到非括号的操作符
    else if symbol is operator:
      while !opStack.empty          and 
            opStack.top is operator and
            opStack.top.precedency >= symbol.precedency:
        if symbol is not bracket:
          opStack.push(symbol)

    # 遇到操作数或字符
    else:
      postfix.push(symbol)

  return postfix

\end{verbatim}

可以看到,在转换的过程中有多个情况处理。当遇到正则表达式的末尾时,应当把当前所有栈中残留的操作符都顺序抛出,以确保不会有操作符遗留在操作符栈中。当遇到左括号的时候,现将其压进栈中,等待与右括号的匹配。当遇到右括号的时候,应该抛出与它匹配的左括号与它之间的所有操作符,这相当于处理完一个组。当遇到其他操作符的时候,根据优先级进行处理,如果栈中有优先级比当前运算符高的,将其抛出,这避免了二义性,抛出完以后,再将当前的操作符压栈,这样也确保如果将来有优先级比它高的运算符,可以在更位于栈顶的地方。如果遇到的是操作数,那么可以直接将其输出到后续表达式中。


\section{状态机图结构}

理论上在得到后序正则表达式以后即可进行NFA的构建,但在此先用一个小节来谈谈NFA的结构。NFA与DFA实际上都是图,其状态对应图的节点,其状态的转移对应于图的边,所以必须确定一种图结构来描述状态机,才能够使得状态机的建立尽可能的方便。

考虑NFA,每个状态用状态编号唯一标记,并且可以作为多条边的起始节点;对于状态的转移,可以用一个三元组来标记,即起始状态,触发转移的输入符号,以及结束状态。这样已经可以初步确立状态以及边的数据结构,看下面的伪代码:

\begin{verbatim}

NFAState:
  id    -> Int            # 唯一标识状态的整型
  edges -> Array(NFAEdge) # 以该状态为起始状态的边的集合

NFAEdge:
  prev  -> Int  # 起始状态的id
  next  -> Int  # 结束状态的id
  input -> Char # 触发转移的输入字符

\end{verbatim}

这里要辨析一下起始状态以及结束状态的概念。这里的起始状态并不是指整个状态机的起始状态,结束状态亦然。这两个概念仅仅针对于状态机中的一条边所关联的两个状态而已。既然状态节点以及状态转移的边的数据结构都有了,那么很自然的就可以得出图的结构:

\begin{verbatim}

Graph:
  begin  -> NFAState        # 图的开始状态
  end    -> Array(NFAState) # 图的结束状态集合
  states -> Array(NFAState) # 图的状态集合

\end{verbatim}

图的开始状态有且仅有一个,而结束状态可以有多个,从上面可以看出,图的结构确实可以非常简单。如果要将这些子图连接在一起,那么可以为两个图之间的某些状态添加边。但这里涉及到一个状态编号唯一性的问题。如果有两个图,其中一个状态编号为$1$到$9$,而另一个图的状态编号为$1$到$7$,那么这两个图合并以后,状态的编号就不唯一了,显然这是不行的。

解决的办法也很容易,我另外写了一个$ID$分配器,这个分配器是将标识号递增分配的,所以每次获得的标识号都不一样。当然,前提是每次分配都使用同一个分配器的实例。这个分配器代码简单,贴上来也不妨:

\begin{verbatim}

function IDGenerator(begin) {
  this.id = begin || 0;
}

IDGenerator.prototype.nextID = function() {
  return this.id++;
} 

IDGenerator.prototype.currID = function() {
  return this.id;
} 

IDGenerator.prototype.resetID = function() {
  this.id = 0;
}

\end{verbatim}

至此以万事俱备,可以进入到NFA的构建了。


\section{$NFA$的构建}

对于NFA的构建,我才用的是效率较低,但非常直观的$McMaughton$-$Yamada$-$Thompson$算法。这种算法是一个递归的构建NFA方法,从细致处出发,慢慢构建成庞大的NFA。由于书中对算法进行了详细的描述,这里就不细说了。下面谈谈实现的问题。先看看伪代码:

\begin{verbatim}

function NFAConstruction(beginID, postfix):
  nfa1, nfa2, begin, end, edge
  nfaStack = {}
  id = new IDGenerator(begin || 0)

  for symbol in postfix:
    begin = new NFAState(id.next)
    end   = new NFAState(id.next)

    if symbol is not operator:
      begin.addEdge(begin, end, symbol)

    else:
      switch symbol:
        case ``|'': generate ``|'' NFA
        case ``~'': generate ``~'' NFA
        case ``*'': generate ``*'' NFA

    nfaStack.push(begin, end)

\end{verbatim}

首先这一个NFA的构建函数接受两个参数(真正的代码会有所不同),其中一个是$beginID$,另一个是$postfix$。$beginID$在现阶段是没有作用的,等到后面构建词法分析器的时候,需要将各个正则表达式的NFA合并的时候,这个参数能够指明在NFA构建的时候,选取的第一个标识号为多少,通过这样指定,能够使得不同NFA里面的每个状态的标识号都不会有重复。至于第二个参数,即后序的正则表达式。

整个算法对后序正则表达式里面的每个符号进行扫描,在每一轮循环里面,先判断符号是不是操作符,即$\sim$,$|$或者$*$,如果不是的话,那么直接为开始状态添加一条以该符号作为触发符号的转移边。如果当前符号是操作符的话,那么就意味着此时需要对过去的子NFA进行组合构建了。这样的子NFA合并跟操作符的操作数有关,对于$|$和$\sim$,会需要到$nfa1$和$nfa2$,而由于$*$是一元操作符,所以只需要到$nfa1$。

上面对子NFA的组合进行了非常简略的描述,仅仅用了伪代码“generate symbol NFA”来表示,下面以操作符“$|$”为例阐述一下合并的过程。

\pagestyle{empty}
%
%
%

\enlargethispage{100cm}
% Start of code
% \begin{tikzpicture}[anchor=mid,>=latex',line join=bevel,]
\begin{tikzpicture}[>=latex',line join=bevel,]
  \pgfsetlinewidth{1bp}
%%
\pgfsetcolor{black}
  % Edge: i -> begin_N_s
  \draw [->] (52.737bp,50.66bp) .. controls (65.036bp,53.474bp) and (80.376bp,56.984bp)  .. (105.04bp,62.628bp);
  \definecolor{strokecol}{rgb}{0.0,0.0,0.0};
  \pgfsetstrokecolor{strokecol}
  \draw (76bp,63.5bp) node {$\epsilon$};
  % Edge: begin_N_t -> end_N_t
  \draw [->] (195.36bp,18bp) .. controls (212.27bp,18bp) and (231.37bp,18bp)  .. (258.92bp,18bp);
  \draw (227bp,25.5bp) node {$N(t)$};
  % Edge: end_N_s -> f
  \draw [->] (337.48bp,63.396bp) .. controls (350.3bp,60.373bp) and (364.55bp,57.012bp)  .. (386.54bp,51.826bp);
  \draw (364bp,65.5bp) node {$\epsilon$};
  % Edge: i -> begin_N_t
  \draw [->] (52.737bp,39.34bp) .. controls (65.146bp,36.5bp) and (80.651bp,32.953bp)  .. (105.51bp,27.266bp);
  \draw (76bp,41.5bp) node {$\epsilon$};
  % Edge: end_N_t -> f
  \draw [->] (335.12bp,27.335bp) .. controls (348.49bp,30.994bp) and (363.71bp,35.156bp)  .. (386.54bp,41.402bp);
  \draw (364bp,42.5bp) node {$\epsilon$};
  % Edge: begin_N_s -> end_N_s
  \draw [->] (195.78bp,72bp) .. controls (212.23bp,72bp) and (230.73bp,72bp)  .. (257.64bp,72bp);
  \draw (227bp,79.5bp) node {$N(s)$};
  % Node: f
\begin{scope}
  \definecolor{strokecol}{rgb}{0.0,0.0,0.0};
  \pgfsetstrokecolor{strokecol}
  \draw (408bp,47bp) ellipse (18bp and 18bp);
  \draw (408bp,47bp) ellipse (22bp and 22bp);
  \draw (408bp,47bp) node {$f$};
\end{scope}
  % Node: i
\begin{scope}
  \definecolor{strokecol}{rgb}{0.0,0.0,0.0};
  \pgfsetstrokecolor{strokecol}
  \draw (27bp,45bp) ellipse (27bp and 18bp);
  \draw (27bp,45bp) node {$i$};
\end{scope}
  % Node: begin_N_t
\begin{scope}
  \definecolor{strokecol}{rgb}{0.0,0.0,0.0};
  \pgfsetstrokecolor{strokecol}
  \draw (147bp,18bp) ellipse (48bp and 18bp);
  \draw (147bp,18bp) node {$begin[N(t)]$};
\end{scope}
  % Node: begin_N_s
\begin{scope}
  \definecolor{strokecol}{rgb}{0.0,0.0,0.0};
  \pgfsetstrokecolor{strokecol}
  \draw (147bp,72bp) ellipse (49bp and 18bp);
  \draw (147bp,72bp) node {$begin[N(s)]$};
\end{scope}
  % Node: end_N_s
\begin{scope}
  \definecolor{strokecol}{rgb}{0.0,0.0,0.0};
  \pgfsetstrokecolor{strokecol}
  \draw (300bp,72bp) ellipse (42bp and 18bp);
  \draw (300bp,72bp) node {$end[N(s)]$};
\end{scope}
  % Node: end_N_t
\begin{scope}
  \definecolor{strokecol}{rgb}{0.0,0.0,0.0};
  \pgfsetstrokecolor{strokecol}
  \draw (300bp,18bp) ellipse (41bp and 18bp);
  \draw (300bp,18bp) node {$end[N(t)]$};
\end{scope}
%
\end{tikzpicture}
% End of code



从图中可以看出,其中$N(s)$和$N(t)$都是子NFA,在这一步中,需要添加两个状态,一个为$i$,另一个为$f$,并且需要添加四条边,触发符号都是$\epsilon$。四条边的起始状态,结束状态分别为:

\begin{center}
\begin{eqnarray*}
     i & \rightarrow & begin[N(s)] \\
     i & \rightarrow & begin[N(t)] \\
     f & \rightarrow & end[N(s)] \\
     f & \rightarrow & end[N(t)] 
\end{eqnarray*}
\end{center}

\newpage

那么怎样才可以拿到那两个子NFA呢?上面的NFA构造的伪代码中,有一个栈,称为“nfaStack”,这个栈记录的是子NFA,每次遇到这些操作符的时候,可以根据操作符的元数来取出相应个数的子NFA,所以只要取出栈顶的两个子NFA即可。这里还需要注意一个小小的问题,那就是取出的顺序。我们可以从减法中看到教训,如果取出的两个操作数顺序不对,被减数与减数的位置就会被调换。虽然“$|$”和“$\sim$”操作符都是满足交换律,但是为了严谨,我们还是搞清楚这个顺序比较好。栈顶的元素作为\begin{bfseries}第二个子NFA\end{bfseries}。看到如下合并的伪代码:

\begin{verbatim}

case ``|'':
  nfa2 = nfaStack.pop
  nfa1 = nfaStack.pop
  begin.addEdge(begin, nfa1.begin, epsilon)
  begin.addEdge(begin, nfa2.begin, epsilon)
  nfa1.end.addEdge(nfa1.end, end, epsilon)
  nfa2.end.addEdge(nfa2.end, end, epsilon)
  break

\end{verbatim}

其中代码的前两句是取出栈顶的两个子NFA,接下来四句创建上面说到的四条新的状态转移边,而两个新的状态$begin$和$end$则是在循环开始的地方建立的。具体可以看上面NFA构建的伪代码。


\section{$\epsilon$闭包与$Move$函数}

在NFA里面有两个重要的函数,它们是$\epsilon$闭包函数以及$Move$函数,其重要性与语法分析中的$Closure$函数和$Goto$函数相当,而且我们也能从其相似性再一次看出词法分析与语法分析工作的层次有何异同。可以说,$\epsilon$闭包与$Closure$函数相互对应,$Move$函数与$Goto$函数相互对应。其实有一种理解方式,就是NFA分析输入流的过程可以看作是一个变种的BFS,每一次只扩散一层,而在这里,某一步通过$\epsilon$能够到达的地方组成一层,这样每一次迭代检查最外层里面有没有结束状态,如果有,NFA就接受输入流了。

由于这两个函数在书中都给出了详细的伪代码,并且在实现上并没有太大的难度,所以这里就不继续细说了。

\section{$DFA$的构建}

在前面说过,DFA可以根据正则表达式的语法分析树直接构建,正则表达式的语法分析树我有实现,但是并没有通过这种方式来构建DFA,而是直接根据NFA来构建DFA,这样会来得更加的清晰以及简明。

由于DFA的每个状态都是NFA里面几个状态的集合,同时,DFA在本质上结构与NFA一致,所以在数据结构上不需要做什么修改,但必须有一种新的命名方式来使得DFA里面的每个状态都有唯一的标识。有一种异常简单的方法就是,将NFA的状态集里面的元素连接起来作为DFA的一个状态的标识。比如NFA里面,状态集${1,2,3}$是DFA里面的一个状态,那么我们称DFA里面的这个状态就叫做$123$,这样确保没有重复。再有一个小问题,那就是,如果按照NFA的状态集里面的元素本来的顺序进行连接的话,又会出现新的问题,这样会导致$123$与$213$为两个不同的DFA状态,但实际上他们都是同一个状态。所以在命名之前必须对状态集里面的元素进行排序。

为了应对上面的情况,我使用另外的一种$ID$产生器,我将此称为\begin{bfseries}名字产生器\end{bfseries},它会将传入的名字转化为一个独一无二的整形$ID$。在进一步阐述之前可以先看一下下面的步骤:

\begin{center}\begin{tabular}{c|ccc}
          & \bf{input} &             & \bf{id} \\ \hline
    $(1)$ & $A$        & $\rightarrow$ & $1$   \\
    $(2)$ & $B$        & $\rightarrow$ & $2$   \\
    $(3)$ & $A$        & $\rightarrow$ & $-1$  \\
    $(4)$ & $C$        & $\rightarrow$ & $3$   
\end{tabular}\end{center}

为什么要将字符串名字转化为整形$ID$呢?因为用整形$ID$会更为方便简洁。从上面的行为我们可以知道,当我们把名字传入这个名字产生器的时候,它会产生出一个整形的$ID$,如果传入的名字已经在之前出现过了,那么将返回一个$-1$。这样,我们就可以把DFA那些冗长的$ID$转化为简洁的整型了。

从以上所需求的行为,我们可以很容易写出名字产生器的数据结构:

\begin{verbatim}

NameID:
  id    -> Int           # 最新分配出去的整型ID
  names -> Array(String) # 记录所有出现过的字符串名字

\end{verbatim}
                  
通过这个简单的数据结构即可实现上面的功能。当每次需要产生标识的时候,检查传入的名字在之前是否已经出现过了,如果已经出现过了,则返回$-1$,如果传入的是一个新的名字,那么则更新最新分配出去的$ID$并且返回这个新的$ID$。

解决了这个问题之后,后面的问题都将非常容易解决。书中给出了构建DFA的伪代码,将其翻译为程序语言并不难,只需要在新状态产生的时候调用上面的名字生成器产生标识就可以了。

\section{一个简易的词法分析器}

这个简易的词法分析器的工作步骤跟本章一开头所提出的步骤是一模一样的,下面从跟用户的交互过程进一步说说这个步骤:

\begin{enumerate}
    \item 用户输入描述词法的正则表达式群。
    \item 用户按照一定格式输入每个正则表达式识别出来后所对应的动作。
    \item 词法分析器将每个输入的正则表达式转换为后序形式。
    \item 词法分析器为每个后序正则表达式建立NFA。
    \item 词法分析器将这些得到的NFA都进行合并,并产生出一个词法NFA。
    \item 词法分析器将这个统一的词法NFA转换为DFA。
    \item 词法分析器将这个DFA的每个接受状态跟其对应的动作关联起来。
    \item 用户输入所要识别的字符串。
    \item 词法分析器对字符串进行分析,输出词法单元流。
\end{enumerate}

为了方便处理,我规定了一种输入格式,当然,这种格式有很大程度上是参考$lex$的格式,首先看看正则表达式的声明如下:

\begin{verbatim}
  letter -> a | b | c | d
  digit  -> 1 | 2 | 3 | 4
  id     -> letter ( letter | digit ) *
  number -> digit digit *
\end{verbatim}

这样在处理的时候,首先将输入串根据行进行切分,然后根据“->”进行每行的此法单元名称及其正则声明的切割,比如第一行中,被切割为词法名称“$letter$”与正则定义“a|b|c|d”。这个正则定义根据后序转换,建立NFA两个步骤处理,之后我们得到的是下面的一个对应关系:

\[
    TokenName \rightarrow TokenNFA
\]

当每个词法单元的正则声明都进行这样的处理以后,我们可以得到如下的映射关系:

\begin{eqnarray*}
    TokenName_1 & \rightarrow & TokenNFA_1 \\
    TokenName_2 & \rightarrow & TokenNFA_2 \\
                & \vdots      &            \\
    TokenName_n & \rightarrow & TokenNFA_n 
\end{eqnarray*}

紧接着把以上得到的NFA都合并起来。这里的子NFA生成以及合并受到之前讨论的命名问题的限制。还记得之前建造NFA的时候需要传入一个参数,指明NFA内部状态的起始标识号吗?在整个子NFA构建过程中,只使用一个$ID$生成器的实例,然后从$1$开始标识状态,每次新构建NFA的时候,都用前一个NFA的结束号加$1$为新的起始号,这样确保各个NFA的状态标识号是连续的整数,并没有任何重复。可以看以下的状态编号示意:

\begin{eqnarray*}
    TokenName_1 & \rightarrow & TokenNFA_1 (1\cdots id(NFA_1)) \\
    TokenName_2 & \rightarrow & TokenNFA_2 (id(NFA_1)+1\cdots id(NFA_2)) \\
                & \vdots      &            \\
    TokenName_n & \rightarrow & TokenNFA_n (id(NFA_{n-1})+1\cdots id(NFA_n))
\end{eqnarray*}

如此编号完成以后,可以将以上众多NFA都合并起来,这样新的NFA编号就从$1$到$id(NFA_n)$了。当然,在合并以后也是需要将所有的结束状态都记住的,因为每个结束状态对应一个执行动作,这个关系什么情况下都不能丢失。对于正则表达式识别与对应的动作关系的输入,我也指定了输入规则,下面是一个例子:

\begin{verbatim}
  {id}     -> {return(ID);}
  {number} -> {return(NUM);}
  >=       -> {return(RELOP);}
  <=       -> {return(RELOP);}
  =        -> {return(RELOP);}
  >        -> {return(RELOP);}
  <        -> {return(RELOP);}
\end{verbatim}

由于JavaScript不能直接调用代码段,所以我只设定了最简单的返回动作,不过在概念上是通用的,只是我没能够实现而已。可以看出在动作定义里面,词法单元有的在上面已经定义过了,如$id$和$number$,但还有一些是默认的词法单元,如各关系符等。对于非默认的词法单元,直接采用上面根绝其正则定义所得到的NFA,而对于默认的词法单元,则另外根据其字符构建NFA,最后再将这些NFA合并起来即可。其实合并过程也就是添加一个开始状态,我将其的标识号标记为$0$,然后从这个开始状态为各个子NFA的开始状态伸展出边,这些边的触发字符为$\epsilon$,如此即可。然后同时保留着一张像下面给出的表:

\begin{center}
\begin{tabular}{c|c|c}
    \bf{Token} & \bf{End States Set} & \bf{Action} \\ \hline
    $token_1$  & $StateSet_1$        & $action_1$  \\
    $token_2$  & $StateSet_2$        & $action_2$  \\
    \vdots     & \vdots              & \vdots      \\
    $token_n$  & $StateSet_n$        & $action_n$  \\
\end{tabular}
\end{center}

这样只要知道此法单元或者结束状态就能够知道执行什么动作了。但在NFA中有一个重要的问题,那就是最长匹配的问题。在最长匹配里面,每次到达结束状态都不能立刻确定执行什么动作,因为总有可能在后面遇到更长的满足当前输入串的模式,所以需要有一种方法来解决这个问题。

有一个简便的方法可以解决。那就是用两个浮动的指针,一个记录前一次到达状态机结束状态时候,输入串的位置;另一个记录这一次到达状态机结束状态时,输入串的位置。可以用下面的简图来说明这个问题:

\begin{center}
\begin{tabular}{cccccc}
    \hline
    $matched$ \vline & $\cdots$ & \vline $c_m$ \vline & $\cdots$ & \vline $c_n$ \vline & $\cdots$ \\
    \hline
           &  & $\uparrow$ &          & $\uparrow$ &          \\
           &  & $previous$ &          & $current$ &          \\
\end{tabular}
\end{center}

观察上面的图,在$matched$部分,是指已经能够匹配的部分。在$matched$到$previous$部分是指可以匹配但仍没有确定的部分,而在$matched$到$current$部分是新的可以确定的部分。每当出现这样的情况,会更新$previous$指针,使其指向$current$的位置,并继续将$current$向右扫描。直至下一次出现这样的情况或遇到输入串的末端。如果$current$一直扫描到最后都没有能匹配上更长的正则表达式,那么$previous$的部分就被确定下来。可以简单地把$previous$指针看作是一步回退的作用。而在这种情况下,一步回退已经足够了。

当然,以上只是讨论了在实现中一些需要注意的问题,其实整体流程书中都说得很详细,在此就略过了。并且这一部分的代码也很多,所以这里并不贴出,具体请参见代码。代码中有比较详细的注释。
