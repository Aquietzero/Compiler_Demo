\chapter{词法分析}

词法分析跟语法分析有很多相同的地方,工作在不同的抽象层次,所以有很多地方可以进行类比理解。其一:在词法分析中,分析的基本单元是字母表里面的符号;而在语法分析中,分析的基本单元是词法分析所向上提供的词法单元,可以知道,词法单元是由字母表中的字母所组成的,这说明语法分析工作在比词法分析更高的层次上。其二:在词法分析中,对于分析机理的描述采用的是正则表达式,而在语法分析中,对分析机理的描述是采用语法描述,这两者的本质其实都是状态机,词法分析中,状态转移接受字母表的基本符号作为输入,而在语法分析中,状态转移接受词法单元作为输入,它们的本质都是一样的,只不过工作在不同抽象层次上面;其三:在词法分析中,分析的目标是得出词法单元,而在语法分析中,分析的目标是句子,而句子正是由词法单元组成,再一次说明工作所在的抽象层次不一样。由此我们可以对比着对两者进行理解,从而找到其共性。

在词法分析中,输入是字符串,而输出是词法单元流。有两种主要的分析流程可以达到这样的目的,其中第一种为:

\begin{enumerate}
    \item 为中序正则表达式添加连接符。
    \item 将中序正则表达式转化为后序正则表达式。
    \item 根据后序正则表达式建立NFA。
    \item 将NFA转化为DFA。
    \item 将DFA的状态最小化。
\end{enumerate}

第二种方式为:

\begin{enumerate}
    \item 为中序正则表达式添加连接符。
    \item 将中序正则表达式转化为后序正则表达式。
    \item 根据后序正则表达式建立抽象语法树。
    \item 计算抽象语法树的$nullabe$,$firstpos$,$lastpos$以及$followpos$。
    \item 根据得出的数据直接构造DFA。
\end{enumerate}

由上面的步骤可以看出,两种方法都必须要先对正则表达式进行处理,鉴于时间关系,我没能把两种方法都实现,于是只实现了第一种方法。

\section{正则表达式}

正则表达式的实现是一个复杂的问题。在JavaScript里面,有现成的正则表达式工具,但基于我这次项目的目的,我打算自己实现正则表达式。在实现的过程中遇到了不少的问题。

首先,正则表达式分为基本正则表达式以及扩展正则表达式。在不同编程语言里面所提供的正则表达式工具都是强大的扩展正则表达式,其中具有很多的操作符以及诸多强大的功能。而基本的正则表达式只有三种操作:

\begin{center}\begin{tabular}{c|c}
    $\bf{Operations}$ & $\bf{Definitions}$ \\
    \hline
    $Union$          & $L \cup M = \{s|s \in L \| s \in M\}$  \\
    $concatenation$  & $LM = \{st|s \in L \&\& s \in M\}$  \\
    $Kleene closure$ & $L^* = \cup^\infty_{i=0} L^i$  \\
\end{tabular}\end{center}

这三种操作已经可以完全表示扩展表达式中的其他操作,所以为了实现上的简洁,我实现的是基本正则表达式。

在实现过程中遇到的另外一个重要的问题是连接符的问题,由于在输入正则表达式的时候是不会显式输入连接符,这给正则表达式的分析带来了很大的困难。比如考虑正则表达式$(a|b)*abb$,如果直接对其进行中序转后序的操作,我们得到的将会是$ab|*abb$,然而实际上这样做是不行的,因为连接操作并没有在这里体现出来,如果就这样直接进行NFA的构建,那么将会得到一个错误的自动机。解决的办法有两个,一个是在中序正则表达式加上连接符以后再转化为后序正则表达式;二是在中序正则表达式转化为后序正则表达式之后再添加连接符;考虑到括号为我们提供的诸多信息,我的实现是先为中序的正则表达式添加上连接符,如上面的正则表达式,添加上连接符以后变成:$(a|b)*\sim a\sim b\sim b$(这里假设用$\sim$来表示连接符)。这样处理以后,转化出来的后序正则表达式为:$ab|*a\sim b\sim b\sim$,这才是正确的后序转换,其后构建状态机才不会出错。

解决了这个问题以后,下一个问题随即出现:应该在什么地方添加连接符呢?之前正是为了这个原因才选择基本正则表达式的,其可能出现的情况比较少,下面可以来穷举一下几种需要添加连接符的情况:

\begin{eqnarray*}
    Char \quad Char & \rightarrow & Char \sim Char \\
    Char \quad (    & \rightarrow & Char \sim ( \\
    ) \quad Char    & \rightarrow & ) \sim Char \\
    * \quad Char    & \rightarrow & * \sim Char \\
    * \quad (       & \rightarrow & * \sim ( \\
    ) \quad (       & \rightarrow & ) \sim ( \\
    any \quad endmarker & \rightarrow & any \sim endmarker \\
\end{eqnarray*}

上面的情况应该也很容易理解,就拿第一种情况来说,如果两个字母表中的字符粘连在一起,那么它们之间必定有一个连接符,因为没有0元操作符可以连接两个字符。又比如右括号和字符之间必定有连接符,右括号预示着一个部分的结束,它后面除非是结束,不然一定需要一个左结合的运算符将其与后面的部分连接,所以这种情况也必须要添加连接符。其余的如此类推。   

解决了这些问题以后,就可以思考正则表达式的数据结构了。首先正则表达式是一个字符串,故可以用字符串来表示正则表达式,考虑到正则表达式需要时时查询字母表,所以有必要把字母表也记录下来,故正则表达式的数据结构可以简单地表示为:

\begin{verbatim}

ReExpression:
    reExp    -> Array(Char) # 正则表达式
    alphabet -> Array(Char) # 正则表达式所表示语言的字母表

\end{verbatim}

有了正则表达式的基本模型以后,可以根据上面的讨论很轻易地写出添加连接符的$insertConcatenation$函数,其伪代码如下:

\begin{verbatim}

function insertConcatenation:
  left  = reExp[0]
  right = reExp[1]

  while (right != endmarker):
    if left and right satisfy the conditions above:
      reExp.insert('~')

    left moves forward
    right moves forward

\end{verbatim}

这里在实现的时候有一个问题需要注意,那就是插入连接符的动作其实是对原字符串进行了修改,如果整个过程都在原字符串上面操作,那么在每次插入之后都必须显式将下标向前移动,以配合新插入的连接符。如果上面的步骤是在一个全新的字符串上面进行操作,那么就不会出现下标错位的问题。


\section{中序表达式与后序表达式}

在处理了连接符的问题以后,就可以开始考虑怎样把中序表达式转化为后序表达式了。首先必须明白为什么要将中序表达式转化为后序表达式。因为中序表达式具有二义性,而后序表达式没有,再者,在中序表达式中,括号仅仅起着分组的作用,并没有任何的运算功能,所以没有必要去保留。在后序表达式中,运算符均在其相关操作数的后方,所以可以直接根据操作符号的元数来进行操作数个数的选取。对于这一点来说,中序表达式是不能做到的,因为在运算符后方仍有与其相关的操作数。

了解了中序表达式到后序表达式的重要性,我们还得明白整个转换过程的机理。由于中序表达式具有二义性,所以在转换的时候需要作出一些规定。看如下算术表达式:$a + b \times c$,其转化为后序表达式以后可以有两种形式:

\begin{enumerate}
    \item $a b + c \times$
    \item $a b c \times +$
\end{enumerate}

可以看到,对于第一种转换方法,是先将$a$于$b$相加,然后再与$c$相乘,而对于第二种转换方法,则是先将$b$于$c$相乘,然后再与$a$相加,这样就说明了转换为后序表达式的时候会对运算符的优先级有所依赖,由此,在转换的时候需要根据运算符的优先级进行转换,使得转换出来的结果符合我们所需要的语义。那上面的例子来说,即我们需要的结果是第二种转换方式。

根据正则表达式的语义,我们可以人为表示出各个运算符的优先级,具体看下表:

\begin{center}\begin{tabular}{|c|c|}
    \hline
    \bf{Operator} & \bf{Precedency} \\ \hline
    $($      & $0$        \\ \hline
    $)$      & $0$        \\ \hline
    $*$      & $9$        \\ \hline
    $\sim$   & $8$        \\ \hline
    $|$      & $7$        \\ \hline
\end{tabular}\end{center}

其实这个优先级的数字并没有太多的含义,只要数字之间能够体现其优先级的先后关系即可,这里对数字的选取其实也为了后面对更多运算符的添加预留了空间,所以才从7开始。之所以将括号的优先级设到最低,是因为它们仅仅是为了分组而已,并没有运算功能。下面给出中序表达式转成后序表达式的伪代码:

\begin{verbatim}

function toPosfix(precedencyTable, dimensionTable):
  opStack = {}
  postfix = {}

  while True:
    # 正则表达式的末尾
    if symbol is endmarker:
      while opStack is not empty:
        postfix.push(opStack.pop())
      break

    # 遇到左括号
    if symbol is ``('':
      opStack.push(symbol)

    # 遇到右括号
    else if symbol is ``)'':
      while !opStack.empty and opStack.top != ``('':
        postfix.push(opStack.pop)
      opStack.pop

    # 遇到非括号的操作符
    else if symbol is operator:
      while !opStack.empty          and 
            opStack.top is operator and
            opStack.top.precedency >= symbol.precedency:
        if symbol is not bracket:
          opStack.push(symbol)

    # 遇到操作数或字符
    else:
      postfix.push(symbol)

  return postfix

\end{verbatim}

可以看到,在转换的过程中有多个情况处理。当遇到正则表达式的末尾时,应当把当前所有栈中残留的操作符都顺序抛出,以确保不会有操作符遗留在操作符栈中。当遇到左括号的时候,现将其压进栈中,等待与右括号的匹配。当遇到右括号的时候,应该抛出与它匹配的左括号与它之间的所有操作符,这相当于处理完一个组。当遇到其他操作符的时候,根据优先级进行处理,如果栈中有优先级比当前运算符高的,将其抛出,这避免了二义性,抛出完以后,再将当前的操作符压栈,这样也确保如果将来有优先级比它高的运算符,可以在更位于栈顶的地方。如果遇到的是操作数,那么可以直接将其输出到后续表达式中。


\section{状态机图结构}

理论上在得到后序正则表达式以后即可进行NFA的构建,但在此先用一个小节来谈谈NFA的结构。
