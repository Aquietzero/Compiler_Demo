\documentclass[12pt,leqno]{book}
\usepackage{verbatim}
\usepackage{graphics}
\usepackage{color}
\usepackage{url}
%\usepackage{hyperref}
\usepackage{CJK}

%package for graph
\usepackage[x11names,rgb]{xcolor}
\usepackage[utf8]{inputenc}
\usepackage{tikz}
\usetikzlibrary{snakes,arrows,shapes}
\usepackage{amsmath}

% global settings
\parindent 0em
\textwidth 15cm
\textheight 21cm
\oddsidemargin 1cm
\evensidemargin 1cm

\begin{document}
\begin{CJK}{UTF8}{gbsn}

% title
\title{\bf{基于JavaScript的编译算法演示系统}}
\author{招蕴豪}
\maketitle
\newpage

% contents
\pagestyle{plain}
\pagenumbering{roman}
\parskip 0.2em
\renewcommand{\contentsname}{目录}
\addcontentsline{toc}{chapter}{目录}
\tableofcontents

% preface
\parskip 1em
\chapter*{前言}
\addcontentsline{toc}{chapter}{前言}

\section*{对编译原理的一点感觉}
\addcontentsline{toc}{section}{对编译原理的一点感觉}
早在两年前,我听说大三有编译原理的课,于是就毫不犹豫地随着减价流买了《编译原理》这本书。这本书一直安安静静地躺在我的书柜上,我好像从来就没有想过主动去翻阅它,生怕它里面深奥的内容会触动我敏感的神经。到了大三,终于得接受这门课的洗礼,这本尘封两年的龙书终于重见天日。

对于编译原理,我有说不出的喜爱。原因有很多,最主要的还是因为我对语言的喜爱。这里的语言偏向于人类语言,我喜欢倾听不同国家的确所特有的韵律,并喜欢观察其语法特点。纵观多种语言,由于中文基本元素(汉字)很多,所以语法结构是比较混乱的。正是因为我们是中国人,我们没有必要过于严谨地去学习中文语法,所以才没有能体会到对于一个完全不会中文的人接触到中文,会是如何地艰辛。当时我就在想,有没有什么办法能真正系统地将这些语法总结出来呢?对于其它语言,我们都可以在书店看到很多相关的语法书,可以说这些语法书在一定程度上总结了该语言总体的语法特点,但是实际上,那是及其不严谨的语法说明,其中存在着诸多特例,以及二义性的描述。但同时我又想,或许正是人类语言中的那些充满二义性的表达,才真正使得人类语言如此迷人。


语法作为指导语言元素使用者能够尽量准确使用语言元素的规则,在人类学习某一门语言的时候充当着及其重要的作用。使得语言初学者能够将其学到的语言元素组合起来,并且能够指引语言学习者能够以一种宏观的角度来观察语言的结构以及性质。从机器的角度而言,语法充当这类似的作用。所以当我第一次接触跟编译相关概念的时候,我非常惊讶,对机器能完成跟人脑类似的工作感到相当神奇。虽然机器所需要识别的语法结构跟人类语言的结构差异巨大,但能向前迈出这样的一步,实在是不容易啊。当然,机器所做的事情比人脑做的事情还多了一些,那就是在理解输入以后还得将其转换为目标代码。

编译原理的课程带给了我很大的快乐,感谢老师允许我用自己的方式来阐述我对编译原理的喜爱以及理解;再有是感谢为编译理论做出过贡献的为人们,你们所创造出来的知识使我身心愉悦,心情舒畅。

\section*{有关本文档}
\addcontentsline{toc}{section}{有关本文档}

自从大一接触\LaTeX{}之后,我就深深爱上了这个排版系统。无论是我的数学作业还是博客,都是靠它来帮我排版。但由于\LaTeX{}对中文支持比较差,在Archlinux下面配置比较痛苦,所以本来打算用英语来完成这个文档。但考虑到各个方面的原因,还是勉强使用中文来完成。

\newpage

% main content
\pagestyle{plain}
\pagenumbering{arabic}

\chapter{系统概述}

\section{动机}

由于编译原理这门课程非常重要,能从里面学到的思想有很多,例如流水线以及局部处理局部优化等思想,所以我想更认真地去学习这一门课程。但由于课程内容本身非常抽象,所以我并不能够单单靠阅读书本去理解。虽然用纸笔演算可以解决很多问题,但是当我在演算过程中感受到神奇并不能真切地表现出来,以至于过一段时间以后,我依然对书中的算法感到非常模糊。当然,对算法进行演算无疑是有助于理解,但是由于算法太多,并且复杂,所以有必要借助另外的途径来寻求对其的更深刻的理解。

于是我有了一个想法,那就是将书中的算法实实在在地用程序语言来实现,并找到一种能比较好展示算法演算过程的表达方式展现出来。通过形象的过程描述,或许能使我对算法的流程更为深刻,同时,对算法的实现又可以使我对算法有更深刻的理解。我向来是一个希望究其源而不满足于现象的人,为了这样的目标,我开始了这趟旅途。

\section{语言的选择}

既然下定了决心,我就开始考虑实现的语言了。由于我的目标并不是去实现一个实实在在的编译器,也不是为了将这些算法的效率提高多少个数量级,我的目标只是为了用一种形象的算法来展示这些算法,所以我没有使用一些比较低级的语言,比如C或者是C++。因为基于这些语言的高效编译器已经有很多了。考虑到我需要的是语言的展示能力,我决定使用JavaScript来完成这个任务。

用JavaScript有一些好处,那就是JavaScript的表达能力比低级语言要强,这样在我实现的过程中就可以忽略掉一些并不需要在算法中关注的细微的现实问题,比如一些低级数据结构的实现;再有,JavaScript可以跟HTML5/CSS3相结合,使得界面的设计更为的便捷与轻松,这样就可以把经历真正放到了对算法的理解上。

\section{系统结构}

这个编译算法演示系统可以分为两个独立的部分,一是核心算法,一是用户界面。而用户界面又可以分为结构,样式以及行为三个部分。鉴于本份文档的任务所在,用户界面并不会花太多的篇幅进行介绍。

\pagestyle{empty}
%
%
%

\enlargethispage{100cm}
% Start of code
% \begin{tikzpicture}[anchor=mid,>=latex',line join=bevel,]
\begin{tikzpicture}[>=latex',line join=bevel,]
  \pgfsetlinewidth{1bp}
%%
\pgfsetcolor{black}
  % Edge: UI -> CSS
  \draw [->] (296.34bp,55.74bp) .. controls (303.27bp,52.901bp) and (310.94bp,49.791bp)  .. (318bp,47bp) .. controls (332.27bp,41.361bp) and (348.11bp,35.285bp)  .. (371bp,26.615bp);
  \definecolor{strokecol}{rgb}{0.0,0.0,0.0};
  \pgfsetstrokecolor{strokecol}
  \draw (333bp,52.5bp) node {modify};
  % Edge: Algorithm -> UI
  \draw [->] (366.43bp,79.307bp) .. controls (351.95bp,79.574bp) and (333.94bp,79.242bp)  .. (318bp,77bp) .. controls (314.52bp,76.511bp) and (310.93bp,75.831bp)  .. (297.52bp,72.593bp);
  \draw (333bp,84.5bp) node {result};
  % Edge: UI -> Algorithm
  \draw [->] (299.82bp,62.625bp) .. controls (314.1bp,61.819bp) and (332.11bp,61.636bp)  .. (348bp,64bp) .. controls (351.97bp,64.591bp) and (356.08bp,65.455bp)  .. (369.96bp,69.222bp);
  \draw (333bp,69.5bp) node {select};
  % Edge: UI -> HTML
  \draw [->] (257.12bp,50.085bp) .. controls (249.19bp,43.179bp) and (238.85bp,35.697bp)  .. (228bp,32bp) .. controls (215.93bp,27.887bp) and (202.1bp,26.812bp)  .. (179.67bp,27.496bp);
  \draw (213bp,37.5bp) node {modify};
  % Edge: CSS -> HTML
  \draw [->] (367.92bp,17.39bp) .. controls (330.29bp,16.799bp) and (258.61bp,16.755bp)  .. (198bp,23bp) .. controls (195.1bp,23.298bp) and (192.11bp,23.681bp)  .. (179.03bp,25.758bp);
  \draw (273bp,24.5bp) node {stylize};
  % Edge: User -> HTML
  \draw [->] (54.297bp,31bp) .. controls (72.135bp,31bp) and (95.953bp,31bp)  .. (125.74bp,31bp);
  \draw (90bp,36.5bp) node {request};
  % Edge: HTML -> UI
  \draw [->] (178.16bp,37.962bp) .. controls (195.42bp,42.936bp) and (218.88bp,49.695bp)  .. (247.87bp,58.047bp);
  \draw (213bp,57.5bp) node {call};
  % Edge: HTML -> User
  \draw [->] (129.53bp,21.757bp) .. controls (122.73bp,19.394bp) and (115.17bp,17.188bp)  .. (108bp,16bp) .. controls (92.215bp,13.383bp) and (87.785bp,13.383bp)  .. (72bp,16bp) .. controls (68.081bp,16.65bp) and (64.043bp,17.604bp)  .. (50.47bp,21.757bp);
  \draw (90bp,21.5bp) node {response};
  % Node: Algorithm
\begin{scope}
  \definecolor{strokecol}{rgb}{0.0,0.0,0.0};
  \pgfsetstrokecolor{strokecol}
  \draw (395bp,78bp) ellipse (28bp and 18bp);
  \draw (395bp,78bp) node {Algorithm};
\end{scope}
  % Node: HTML
\begin{scope}
  \definecolor{strokecol}{rgb}{0.0,0.0,0.0};
  \pgfsetstrokecolor{strokecol}
  \draw (153bp,31bp) ellipse (27bp and 18bp);
  \draw (153bp,31bp) node {HTML};
\end{scope}
  % Node: UI
\begin{scope}
  \definecolor{strokecol}{rgb}{0.0,0.0,0.0};
  \pgfsetstrokecolor{strokecol}
  \draw (273bp,65bp) ellipse (27bp and 18bp);
  \draw (273bp,65bp) node {UI};
\end{scope}
  % Node: User
\begin{scope}
  \definecolor{strokecol}{rgb}{0.0,0.0,0.0};
  \pgfsetstrokecolor{strokecol}
  \draw (27bp,31bp) ellipse (27bp and 18bp);
  \draw (27bp,31bp) node {User};
\end{scope}
  % Node: CSS
\begin{scope}
  \definecolor{strokecol}{rgb}{0.0,0.0,0.0};
  \pgfsetstrokecolor{strokecol}
  \draw (395bp,18bp) ellipse (27bp and 18bp);
  \draw (395bp,18bp) node {CSS};
\end{scope}
%
\end{tikzpicture}
% End of code



观察上图。

\chapter{语法分析}

语法分析是编译流水线的第二个部分,语法分析器接受词法分析器所提供的词法单元流,根据给定的语法判断词法单元流是否符合语法。语法分析有两种主要的方法,一种是自顶向下语法分析,另一种是自底向上语法分析。在自顶向下语法分析中,语法分析器从语法的开始符号出发,构造一棵词法单元流的语法分析树;而在自底向上的语法分析中,语法分析器根据移入归约原则,将词法单元流转化为语法的初始符号,其过程中的每一步都对应着该词法单元流最右推导的中间过程。

在语法分析部分,对于自顶向下的语法分析,我实现了$LL(1)$预测分析;对于自底向上的语法分析,我实现了$SLR$以及$LR(1)$。在这些方法的实现过程中,需要到很多的辅助函数,而这些辅助函数都起到了至关重要的作用,下面结合我对语法分析的理解,逐一介绍它们的实现过程。

\section{语法模型}

在语法分析的实现过程中,首当其冲的问题就是为语法选择一个合适的数据结构,一个高效的数据结构非常重要,但高效的数据结构同时又可能难以令人理解。为了均衡高效以及良好阅读性的矛盾,数据结构必须仔细地进行设计。由于JavaScript并没有提供什么数据结构,所以必须自己根据需要来实现。同时,由于数据结构是根据自身来进行实现的,所以比较灵活。下面围绕着书中的表达式文法作为例子,来阐述我的设计过程。一个具有非左递归的文法如下:

\begin{eqnarray*}
    E  & \rightarrow & TE'               \\
    E' & \rightarrow & +TE' | \epsilon   \\
    T  & \rightarrow & FT'               \\
    T' & \rightarrow & *FT' | \epsilon   \\
    F  & \rightarrow & (E) | id          \\
\end{eqnarray*}

从上面的语法可以看出,语法最直观的一个模型就是数组,或者是链表,其基本元素是产生式。其实在JavaScript里面,只提供了对象以及数组两个比较高级的数据结构,其实数组也是对象,所以在我的整个系统里面,数据结构基本上都是根据简单的数组组合构建而成的。既然语法是一个产生式的数组,那么产生式又应该如何表示呢?观察下面的产生式:

\begin{eqnarray*}
    \underbrace{E'}_\text{产生式头} \rightarrow 
    \underbrace{+TE'}_\text{产生式体1} |
    \underbrace{\epsilon}_\text{产生式体2}
\end{eqnarray*}

可以观察到产生式由两个部分组成,一个是产生式头,另一个是产生式体,而由于一个产生式可以有多个产生式体,所以可以用数组来存放一个产生式的所有产生式体。在这里有一个问题是需要注意的,那就是\begin{bfseries}在这里假定每个非终结符号对应一个产生式的数据结构。\end{bfseries}简单地说,就是不会出现下面的结构:

\begin{eqnarray*}
    E' & \rightarrow & +TE' \\
    E' & \rightarrow & \epsilon \\
\end{eqnarray*}

当然,这也是一种合法的语法表示形式,是没有理由禁止的,但为了处理的方便,当用户以这样的方式进行输入的时候,一个称为$reduceRedundancy$的函数会消除这种冗余,即将这样的情况转化为上面的那种用``$|$''来表示的形式。

当然,上面的描述足以表示产生式以及语法两个抽象概念,但是考虑到运算时有一些额外的量是需要的,将这些量绑定到这两个数据结构有助于提高运算效率。例如,在求$First$和$Follow$集的时候,需要查看语法中的终结符号,所以有必要把终结符集合也绑定到语法的数据结构中;又如,在计算预测分析表的时候,需要随时用到某一个非终结符的$First$和$Follow$集的结果,所以把这两个集合也绑定到产生式中也是非常重要的。基于这些考虑,可以得到下面的语法数据结构和产生式数据结构的伪代码。

\begin{verbatim}
    Grammar:
        terminals   -> Array(String)     # 终结符集合
        productions -> Array(Production) # 产生式集合

    Production:
        head   -> Char            # 产生式头
        bodies -> Array(String)   # 产生式体集合
        first  -> Array(String)   # First集
        follow -> Array(String)   # Follow集
\end{verbatim}

\section{消除冗余以及消除左递归}

只要有了上面的模型,即有了解决语法分析问题的工具。我们首要面临的问题就是解决上面所遇到的基本假设,那就是将要进行语法分析的的语法没有冗余,对于自顶向下分析的语法没有左递归。对于消除冗余的问题,解决办法是比较容易的,只要顺序扫描,对于每个产生式头,都检查前面有没有出现过相同的产生式头,如果存在相同的产生式头,将其合并。这样即可消除冗余。最后产生的语法的产生式将是没有重复产生式头的。对于消除左递归,必须先了解下面的立即左递归的消除原理。假设有如下的产生式:

$$
    A \rightarrow A \alpha | \beta
$$

对于这样的左递归式,有既定的模式,即递归产生式体以及递归终结符。作为合法的文法,必定要有这两个部分,如果没有了递归部分,那么就不算是递归文法了,如果没有递归终结符,那文法将无法终止,也是不合适的,所以上面的模式描述了递归文法的通用特点。当用这样的文法对句子“$\beta \alpha \cdots \alpha$”进行分析的时候,它会尽量展开,一直展开知道最后遇到“$\beta$”,所以说这样的分析是无法预计结束点的,所以得尽早将确定的部分先进行代入解决,这就需要左递归的消除了。\begin{bfseries}其实在我看来,对左递归的消除就是将左递归转化为右递归。\end{bfseries}正是因为语法分析过程是对输入串的从左向右的扫描过程,所以可以正常处理右递归,只要句子是有限的,那么递归就一定可以结束。为了转化为右递归,我们先把终结符号先分析出来,于是得到下面的转换:

\begin{eqnarray*}
    A  & \rightarrow & \beta A' \\
    A' & \rightarrow & \alpha A' | \epsilon \\
\end{eqnarray*}

将这种情况推广一下,即可得到一种比较通用的情况,即将:

$$
    A \rightarrow A \alpha_1 | A \alpha_2 | \cdots | A \alpha_m 
                  | \beta_1  | \beta_2    | \cdots | \beta_n
$$

转化为:

\begin{eqnarray*}
    A  & \rightarrow & \beta_1 A'| \beta_2 A' | \cdots | \beta_n A' \\
    A' & \rightarrow & \alpha_1 A' | \alpha_2 A' | \cdots | \alpha_m A' | \epsilon \\
\end{eqnarray*}

这样的转化其实在代码上面也不是那么容易实现的,不过主要思想还是一致的。即先对所有产生式进行扫描,对于其中的某一个产生式,再对出现在这一产生式前面的所有产生式进行扫描,用上面给出的通用方法进行左递归的消除,这样等于两次循环递增消除文法中的左递归。当然,在计算的时候有必要将可以用前面产生式替换的后面非终结符号替换掉,这样保证不会产生新的左递归。

\section{First集与Follow集}

毫无疑问,对于文法非终结符号的$First$和$Follow$集的计算属于最基本的计算,所以处理好这两个算法非常重要。这两个函数其实是独立于文法的,即使文法是左递归的,这两个集合也应当可以正常地算出来。

$Fisrt$集被定义为可以从$\alpha$推导得到的串的首符号的集合,其中$\alpha$为任意文法串。$First$集的作用可以生动地从其定义看出,即对将要进行分析的动向进行一个判断,这些书中都有描述,这里不再详述。对于$First$集合的计算方式也是递归的,计算的时候有以下三种情况:

\begin{enumerate}
    \item 如果$X$是一个终结符号,那么$First(X)=X$。
    \item 如果$X$是一个非终结符号,且$X \rightarrow Y_1 Y_2 \cdots Y_k$是一个产生式,其中$k \leq 1$,那么如果对于某个$i$,$a$在$First(Y_i)$中且$\epsilon$在所有的$First(Y_1),\cdots,First(Y_{i-1})$中,就把$a$加入到$First(X)$中。
    \item 如果$X \rightarrow \epsilon$是一个产生式,那么将$\epsilon$也加入到$First(X)$中。
\end{enumerate}

从上面的$First$集计算过程可以看出,这个计算过程是递归的,所以不可能直接就可以按着这个计算步骤进行计算。可以看看下面的特例:

\begin{eqnarray*}
    A & \rightarrow & Ba \quad | \quad \cdots \\
    B & \rightarrow & Cb \quad | \quad \cdots \\
    C & \rightarrow & Ac \quad | \quad \cdots \\
\end{eqnarray*}

从这个例子结合上面的算法可以看出,如果要计算$First(A)$,那首先得计算$First(B)$,而要计算$Fisrt(B)$,又首先得计算$First(C)$,于是在计算$First(C)$的时候就回归到最开始的问题了,所以不能写一个递归的程序并期待它能自动地给与我们答案。对此必须想一个办法来检测这种“有环”文法。解决办法也很简单,即在每次进行递归的时候,都将当前的状态放到一个寄存状态的地方,而这个地方是可供每一层的递归可见,实现这样的功能可以靠一个简单的状态数组实现。每次将需要求$First$集的非终结符添加进数组,并在每次递归开始之初检测状态数组,看看数组里面有没有这个状态,如果这个状态已经在数组里面了,那说明环已经出现了。要解除这个环的简单方法就是放弃这一次的计算。比如上面的情况,可以直接放弃最后一条产生式的考察,也可以放弃最初要求的计算,因为出现了环,所以只要一端解锁,再算环另一头的时候,就肯定会回到这一头,所以根本不会错过什么计算。

\newpage


\chapter{词法分析}

词法分析跟语法分析有很多相同的地方,工作在不同的抽象层次,所以有很多地方可以进行类比理解。其一:在词法分析中,分析的基本单元是字母表里面的符号;而在语法分析中,分析的基本单元是词法分析所向上提供的词法单元,可以知道,词法单元是由字母表中的字母所组成的,这说明语法分析工作在比词法分析更高的层次上。其二:在词法分析中,对于分析机理的描述采用的是正则表达式,而在语法分析中,对分析机理的描述是采用语法描述,这两者的本质其实都是状态机,词法分析中,状态转移接受字母表的基本符号作为输入,而在语法分析中,状态转移接受词法单元作为输入,它们的本质都是一样的,只不过工作在不同抽象层次上面;其三:在词法分析中,分析的目标是得出词法单元,而在语法分析中,分析的目标是句子,而句子正是由词法单元组成,再一次说明工作所在的抽象层次不一样。由此我们可以对比着对两者进行理解,从而找到其共性。

在词法分析中,输入是字符串,而输出是词法单元流。有两种主要的分析流程可以达到这样的目的,其中第一种为:

\begin{enumerate}
    \item 为中序正则表达式添加连接符。
    \item 将中序正则表达式转化为后序正则表达式。
    \item 根据后序正则表达式建立NFA。
    \item 将NFA转化为DFA。
    \item 将DFA的状态最小化。
\end{enumerate}

第二种方式为:

\begin{enumerate}
    \item 为中序正则表达式添加连接符。
    \item 将中序正则表达式转化为后序正则表达式。
    \item 根据后序正则表达式建立抽象语法树。
    \item 计算抽象语法树的$nullabe$,$firstpos$,$lastpos$以及$followpos$。
    \item 根据得出的数据直接构造DFA。
\end{enumerate}

由上面的步骤可以看出,两种方法都必须要先对正则表达式进行处理,鉴于时间关系,我没能把两种方法都实现,于是只实现了第一种方法。

\section{正则表达式}

正则表达式的实现是一个复杂的问题。在JavaScript里面,有现成的正则表达式工具,但基于我这次项目的目的,我打算自己实现正则表达式。在实现的过程中遇到了不少的问题。

首先,正则表达式分为基本正则表达式以及扩展正则表达式。在不同编程语言里面所提供的正则表达式工具都是强大的扩展正则表达式,其中具有很多的操作符以及诸多强大的功能。而基本的正则表达式只有三种操作:

\begin{tabular}{l|r}
    $Operations$     & $Definitions\quad and\quad Demostration$ \\
    \hline
    $Union$          & $L \cup M = \{s|s \in L \| s \in M\}$  \\
    $concatenation$  & $LM = \{st|s \in L \&\& s \in M\}$  \\
    $Kleene closure$ & $L^* = \cup^\infty_{i=0} L^i$  \\
\end{tabular}

这三种操作已经可以完全表示扩展表达式中的其他操作,所以为了实现上的简洁,我实现的是基本正则表达式。

在实现过程中遇到的另外一个重要的问题是连接符的问题,由于在输入正则表达式的时候是不会显式输入连接符,这给正则表达式的分析带来了很大的困难。比如考虑正则表达式$(a|b)*abb$,如果直接对其进行中序转后序的操作,我们得到的将会是$ab|*abb$,然而实际上这样做是不行的,因为连接操作并没有在这里体现出来,如果就这样直接进行NFA的构建,那么将会得到一个错误的自动机。解决的办法有两个,一个是在中序正则表达式加上连接符以后再转化为后序正则表达式;二是在中序正则表达式转化为后序正则表达式之后再添加连接符;考虑到括号为我们提供的诸多信息,我的实现是先为中序的正则表达式添加上连接符,如上面的正则表达式,添加上连接符以后变成:$(a|b)*\sim a\sim b\sim b$(这里假设用$\sim$来表示连接符)。这样处理以后,转化出来的后序正则表达式为:$ab|*a\sim b\sim b\sim$,这才是正确的后序转换,其后构建状态机才不会出错。

解决了这个问题以后,下一个问题随即出现:应该在什么地方添加连接符呢?之前正是为了这个原因才选择基本正则表达式的,其可能出现的情况比较少,下面可以来穷举一下几种需要添加连接符的情况:

\begin{eqnarray*}
    Char \quad Char & \rightarrow & Char \sim Char \\
    Char \quad (    & \rightarrow & Char \sim ( \\
    ) \quad Char    & \rightarrow & ) \sim Char \\
    * \quad Char    & \rightarrow & * \sim Char \\
    * \quad (       & \rightarrow & * \sim ( \\
    ) \quad (       & \rightarrow & ) \sim ( \\
    any \quad endmarker & \rightarrow & any \sim endmarker \\
\end{eqnarray*}

上面的情况应该也很容易理解,就拿第一种情况来说,如果两个字母表中的字符粘连在一起,那么它们之间必定有一个连接符,因为没有0元操作符可以连接两个字符。又比如右括号和字符之间必定有连接符,右括号预示着一个部分的结束,它后面除非是结束,不然一定需要一个左结合的运算符将其与后面的部分连接,所以这种情况也必须要添加连接符。其余的如此类推。   

解决了这些问题以后,就可以思考正则表达式的数据结构了。首先正则表达式是一个字符串,故可以用字符串来表示正则表达式,考虑到正则表达式需要时时查询字母表,所以有必要把字母表也记录下来,故正则表达式的数据结构可以简单地表示为:

\begin{verbatim}

ReExpression:
    reExp    -> Array(Char) # 正则表达式
    alphabet -> Array(Char) # 正则表达式所表示语言的字母表

\end{verbatim}

有了正则表达式的基本模型以后,可以根据上面的讨论很轻易地写出添加连接符的$insertConcatenation$函数,其伪代码如下:

\begin{verbatim}

function insertConcatenation:
  left  = reExp[0]
  right = reExp[1]

  while (right != endmarker):
    if left and right satisfy the conditions above:
      reExp.insert('~')

    left moves forward
    right moves forward

\end{verbatim}

这里在实现的时候有一个问题需要注意,那就是插入连接符的动作其实是对原字符串进行了修改,如果整个过程都在原字符串上面操作,那么在每次插入之后都必须显式将下标向前移动,以配合新插入的连接符。如果上面的步骤是在一个全新的字符串上面进行操作,那么就不会出现下标错位的问题。

\chapter{程序运行及代码结构}

整体的代码框架在系统概述里面已经介绍过,这里稍微详细地介绍。本次项目将算法实现,外观呈现以及人机交互作为独立的三个部分,以致模块化非常的明显,下面将结合代码结构来谈一下有关算法以外的问题。具体的代码可以参见我的$Github$,地址为\url{https://github.com/Aquietzero/Compiler_Demo}。

建议看代码的时候最好还是到我的$Github$去看,因为那里结构整洁,操作方便,并且有代码高亮,所以看起来比较舒服。再有,在上面给出的网址靠近底部的README部分,有一个演示的链接,点进去以后即可使用这一个编译算法演示系统。所以最好还是不要在我提交的文件夹里面看代码。

\section{程序运行}

如果想独立运行本系统,那也是可以的。支持本系统的最好的浏览器为$firefox$或者$chrome$,千万不要用$IE$或者$360$浏览器,因为我觉得没有必要为那些浏览器专门写特定的展示代码。用$firefox$或者$chrome$打开目录下面的“index.html”文件即可运行,在操作过程中请遵循系统中的提示。一般情况下,程序都有默认的输入,可以按照默认输入的格式来进行输入。

还要说明一点,有部分功能虽然我实现了,但是并没有提供演示的接口,所以在上面阐述算法的时候就没有说到,例如正则表达式的抽象语法树的构建之类的算法。确实,我也感觉没有做好的东西就没有太大必要去阐述了。

\section{整体代码结构}

先看下面的整体代码结构,以下列出的是主要的代码部分,有一些中间生成文件并没有必要列出。整个系统的主要部分其实只有四个,分别是“algo”,“css”,“ui”以及“doc”。其中前三者的关系可以参见系统概述中给出的关系图,而“doc”正是本文档的源文件。其余的一些文件夹比如“imgs”里面存放的是系统展示页面所需要到的图像,“lib”里面存放的是系统所需要到的库文件,其实也就只有“jquery”,而“fonts”里面的是一些系统需要到的字体。

\scriptsize
\begin{verbatim}
  Compiler_Demo
    |__ algo
    |    |__ array.js, id.js, lib.js
    |    |__ grammar.js, production.js
    |    |__ item.js
    |    |__ ll_1.js, slr.js
    |    |__ re.js, preprocRe.js
    |    |__ nfa.js, dfa.js
    |    |__ lexer.js
    |    |__ toHtml.js
    |    |__ compiler_demo.js
    |    \__ tree.js
    |
    |__ css
    |    |__ welcomePage.css, frame.css
    |    |__ navigationPage.css
    |    |__ inputPage.css, comfirmPage.css
    |    |__ ll_1Page.css, lr_0Page.css, lr_1Page.css
    |    \__ postfixPage.css, nfaPage.css, dfaPage.css, lexerPagee.css
    |
    |__ ui
    |    |__ welcomePage.js, frame.js
    |    |__ navigationPage.js
    |    |__ inputPage.js, comfirmPage.js
    |    |__ ll_1Page.js, lr_0Page.js, lr_1Page.js
    |    \__ postfixPage.js, nfaPage.js, dfaPage.js, lexerPagee.js
    |
    |__ doc
    |    |__ compiler_demo.tex
    |    |__ general_idea.tex
    |    |__ syntax_analysis.tex
    |    |__ lexical_analysis.tex
    |    |__ code_structure.tex
    |    |__ pic_sys_structure.tex
    |    \__ pic_nfa1.tex
    |
    |__ imgs
    |__ fonts
    |__ lib
    |__ index.html
    \__ README.md
\end{verbatim}

\normalsize
看到上面的结构应该也挺一目了然的,每个文件的单独作用在这里就不细说了,因为每个文件的注释都说得非常明白,并且文件的名称也清晰指明了这个文件的作用。

\chapter{总结}

本次项目使我获益匪浅,其实主要说的在前言都说了,这里的总结是为了保证文档的完整性。

其实一开始下定决心要写的时候曾经跟几个同学谈过,不过他们都不愿意跟我合作,怕最后分数会很低。但我却不是这么想。在过去的大一大二里面,一直都是专心听讲,课后完成老师布置的作业,并且感觉好像完成了作业以后就没有太多的时间可以自己做喜欢的事情,并且学了两年的程序设计,好像也没试过写超过五千行的代码。这样感觉不是太好,虽然成绩不差,但是学到的东西感觉也不是很多,并且学了的东西没有得到运用。

在遇到编译原理这门课,前几周我确实听得云里雾里,之前的作业也是感觉挺难下手。并且碰巧在这个学期的$Web2.0$课中接触到了JavaScript,深感用$Web$的技术来做演示实在是合适不过了,于是就琢磨着能不能将抽象的知识变成实在的,容易感知以及理解的展示形式。于是才有了这个念头。但可惜,由于只有我一个人在做,在要兼顾其他科目以及不怠慢对课程的理解上,我还是尽量抽出时间完成了两个模块。其实老师的上课速度比我的编码速度快多了,我很难在跟着课程进度的同时使得代码也赶上课程进度,毕竟虽然老师同意不用我做作业,但我还是得考试的。其实作业我都有做的,因为我编码以后得靠作业题来测试,只是在草稿纸上演算,没有提交罢了。

总的来说,本次项目的基本目的达到了,但我却不太满意。本来想着趁假期再多完成一点的,但由于过年的缘故,传统习俗身不由己,难以抽出时间。于是打算将报告写好,将我之前所做的工作好好总结一番就算了,并且带有熟练使用$LaTeX$的目的,我还是带着颇高的积极性去写报告。但没想到一写就是十几天了,在对$LaTeX$的各种困惑以及搜索中完成了这个报告。其实这个报告也不算详细,原因有二,一是书本上的描述非常详尽,并给出了很多的伪代码,而我是跟着书上的伪代码进行编码,所以没有太多的东西需要阐述,所以我只是挑出一些比较重点的实现细节来说,并在某些地方插入自己的理解。再有,我的代码中已经有了比较详细的注释,感觉如果把那些注释都搬到文档中来,未免有些不妥。故诸多地方都是点到即止罢了。

最后,感谢老师一个学年下来的教导,无论是操作系统课程还是编译原理课程,我都从中获益匪浅,希望之后还能上到万海老师的课程。


\end{CJK}
\end{document}
