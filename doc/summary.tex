\chapter{总结}

本次项目使我获益匪浅,其实主要说的在前言都说了,这里的总结是为了保证文档的完整性。

其实一开始下定决心要写的时候曾经跟几个同学谈过,不过他们都不愿意跟我合作,怕最后分数会很低。但我却不是这么想。在过去的大一大二里面,一直都是专心听讲,课后完成老师布置的作业,并且感觉好像完成了作业以后就没有太多的时间可以自己做喜欢的事情,并且学了两年的程序设计,好像也没试过写超过五千行的代码。这样感觉不是太好,虽然成绩不差,但是学到的东西感觉也不是很多,并且学了的东西没有得到运用。

在遇到编译原理这门课,前几周我确实听得云里雾里,之前的作业也是感觉挺难下手。并且碰巧在这个学期的$Web2.0$课中接触到了JavaScript,深感用$Web$的技术来做演示实在是合适不过了,于是就琢磨着能不能将抽象的知识变成实在的,容易感知以及理解的展示形式。于是才有了这个念头。但可惜,由于只有我一个人在做,在要兼顾其他科目以及不怠慢对课程的理解上,我还是尽量抽出时间完成了两个模块。其实老师的上课速度比我的编码速度快多了,我很难在跟着课程进度的同时使得代码也赶上课程进度,毕竟虽然老师同意不用我做作业,但我还是得考试的。其实作业我都有做的,因为我编码以后得靠作业题来测试,只是在草稿纸上演算,没有提交罢了。

总的来说,本次项目的基本目的达到了,但我却不太满意。本来想着趁假期再多完成一点的,但由于过年的缘故,传统习俗身不由己,难以抽出时间。于是打算将报告写好,将我之前所做的工作好好总结一番就算了,并且带有熟练使用$LaTeX$的目的,我还是带着颇高的积极性去写报告。但没想到一写就是十几天了,在对$LaTeX$的各种困惑以及搜索中完成了这个报告。其实这个报告也不算详细,原因有二,一是书本上的描述非常详尽,并给出了很多的伪代码,而我是跟着书上的伪代码进行编码,所以没有太多的东西需要阐述,所以我只是挑出一些比较重点的实现细节来说,并在某些地方插入自己的理解。再有,我的代码中已经有了比较详细的注释,感觉如果把那些注释都搬到文档中来,未免有些不妥。故诸多地方都是点到即止罢了。

最后,感谢老师一个学年下来的教导,无论是操作系统课程还是编译原理课程,我都从中获益匪浅,希望之后还能上到万海老师的课程。
