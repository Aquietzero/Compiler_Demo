\chapter{系统概述}

\section{动机}

由于编译原理这门课程非常重要,能从里面学到的思想有很多,例如流水线以及局部处理局部优化等思想,所以我想更认真地去学习这一门课程。但由于课程内容本身非常抽象,所以我并不能够单单靠阅读书本去理解。虽然用纸笔演算可以解决很多问题,但是当我在演算过程中感受到神奇并不能真切地表现出来,以至于过一段时间以后,我依然对书中的算法感到非常模糊。当然,对算法进行演算无疑是有助于理解,但是由于算法太多,并且复杂,所以有必要借助另外的途径来寻求对其的更深刻的理解。

于是我有了一个想法,那就是将书中的算法实实在在地用程序语言来实现,并找到一种能比较好展示算法演算过程的表达方式展现出来。通过形象的过程描述,或许能使我对算法的流程更为深刻,同时,对算法的实现又可以使我对算法有更深刻的理解。我向来是一个希望究其源而不满足于现象的人,为了这样的目标,我开始了这趟旅途。

\section{语言的选择}

既然下定了决心,我就开始考虑实现的语言了。由于我的目标并不是去实现一个实实在在的编译器,也不是为了将这些算法的效率提高多少个数量级,我的目标只是为了用一种形象的算法来展示这些算法,所以我没有使用一些比较低级的语言,比如C或者是C++。因为基于这些语言的高效编译器已经有很多了。考虑到我需要的是语言的展示能力,我决定使用JavaScript来完成这个任务。

用JavaScript有一些好处,那就是JavaScript的表达能力比低级语言要强,这样在我实现的过程中就可以忽略掉一些并不需要在算法中关注的细微的现实问题,比如一些低级数据结构的实现;再有,JavaScript可以跟HTML5/CSS3相结合,使得界面的设计更为的便捷与轻松,这样就可以把经历真正放到了对算法的理解上。
