\chapter{语法分析}

语法分析是编译流水线的第二个部分,语法分析器接受词法分析器所提供的词法单元流,根据给定的语法判断词法单元流是否符合语法。语法分析有两种主要的方法,一种是自顶向下语法分析,另一种是自底向上语法分析。在自顶向下语法分析中,语法分析器从语法的开始符号出发,构造一棵词法单元流的语法分析树;而在自底向上的语法分析中,语法分析器根据移入归约原则,将词法单元流转化为语法的初始符号,其过程中的每一步都对应着该词法单元流最右推导的中间过程。

在语法分析部分,对于自顶向下的语法分析,我实现了$LL(1)$预测分析;对于自底向上的语法分析,我实现了$SLR$以及$LR(1)$。在这些方法的实现过程中,需要到很多的辅助函数,而这些辅助函数都起到了至关重要的作用,下面结合我对语法分析的理解,逐一介绍它们的实现过程。

\section{语法模型}

在语法分析的实现过程中,首当其冲的问题就是为语法选择一个合适的数据结构,一个高效的数据结构非常重要,但高效的数据结构同时又可能难以令人理解。为了均衡高效以及良好阅读性的矛盾,数据结构必须仔细地进行设计。由于JavaScript并没有提供什么数据结构,所以必须自己根据需要来实现。同时,由于数据结构是根据自身来进行实现的,所以比较灵活。下面围绕着书中的表达式文法作为例子,来阐述我的设计过程。一个具有非左递归的文法如下:

\begin{eqnarray*}
    E  & \rightarrow & TE'               \\
    E' & \rightarrow & +TE' | \epsilon   \\
    T  & \rightarrow & FT'               \\
    T' & \rightarrow & *FT' | \epsilon   \\
    F  & \rightarrow & (E) | id          \\
\end{eqnarray*}

从上面的语法可以看出,语法最直观的一个模型就是数组,或者是链表,其基本元素是产生式。其实在JavaScript里面,只提供了对象以及数组两个比较高级的数据结构,其实数组也是对象,所以在我的整个系统里面,数据结构基本上都是根据简单的数组组合构建而成的。既然语法是一个产生式的数组,那么产生式又应该如何表示呢?观察下面的产生式:

\begin{eqnarray*}
    \underbrace{E'}_\text{产生式头} \rightarrow 
    \underbrace{+TE'}_\text{产生式体1} |
    \underbrace{\epsilon}_\text{产生式体2}
\end{eqnarray*}

可以观察到产生式由两个部分组成,一个是产生式头,另一个是产生式体,而由于一个产生式可以有多个产生式体,所以可以用数组来存放一个产生式的所有产生式体。在这里有一个问题是需要注意的,那就是\begin{bfseries}在这里假定每个非终结符号对应一个产生式的数据结构。\end{bfseries}简单地说,就是不会出现下面的结构:

\begin{eqnarray*}
    E' & \rightarrow & +TE' \\
    E' & \rightarrow & \epsilon \\
\end{eqnarray*}

当然,这也是一种合法的语法表示形式,是没有理由禁止的,但为了处理的方便,当用户以这样的方式进行输入的时候,一个称为$reduceRedundancy$的函数会消除这种冗余,即将这样的情况转化为上面的那种用``$|$''来表示的形式。

当然,上面的描述足以表示产生式以及语法两个抽象概念,但是考虑到运算时有一些额外的量是需要的,将这些量绑定到这两个数据结构有助于提高运算效率。例如,在求$First$和$Follow$集的时候,需要查看语法中的终结符号,所以有必要把终结符集合也绑定到语法的数据结构中;又如,在计算预测分析表的时候,需要随时用到某一个非终结符的$First$和$Follow$集的结果,所以把这两个集合也绑定到产生式中也是非常重要的。基于这些考虑,可以得到下面的语法数据结构和产生式数据结构的伪代码。

\begin{verbatim}
    Grammar:
        terminals   -> Array(String)     # 终结符集合
        productions -> Array(Production) # 产生式集合

    Production:
        head   -> Char            # 产生式头
        bodies -> Array(String)   # 产生式体集合
        first  -> Array(String)   # First集
        follow -> Array(String)   # Follow集
\end{verbatim}

\newpage

